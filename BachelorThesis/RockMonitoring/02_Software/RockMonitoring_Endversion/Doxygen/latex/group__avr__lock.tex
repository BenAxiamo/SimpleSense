\hypertarget{group__avr__lock}{}\section{$<$avr/lock.h$>$\+: Lockbit Support}
\label{group__avr__lock}\index{$<$avr/lock.\+h$>$\+: Lockbit Support@{$<$avr/lock.\+h$>$\+: Lockbit Support}}
\begin{DoxyParagraph}{Introduction}

\end{DoxyParagraph}
The Lockbit A\+PI allows a user to specify the lockbit settings for the specific A\+VR device they are compiling for. These lockbit settings will be placed in a special section in the E\+LF output file, after linking.

Programming tools can take advantage of the lockbit information embedded in the E\+LF file, by extracting this information and determining if the lockbits need to be programmed after programming the Flash and E\+E\+P\+R\+OM memories. This also allows a single E\+LF file to contain all the information needed to program an A\+VR.

To use the Lockbit A\+PI, include the $<$\hyperlink{io_8h}{avr/io.\+h}$>$ header file, which in turn automatically includes the individual I/O header file and the $<$\hyperlink{lock_8h}{avr/lock.\+h}$>$ file. These other two files provides everything necessary to set the A\+VR lockbits.

\begin{DoxyParagraph}{Lockbit A\+PI}

\end{DoxyParagraph}
Each I/O header file may define up to 3 macros that controls what kinds of lockbits are available to the user.

If \+\_\+\+\_\+\+L\+O\+C\+K\+\_\+\+B\+I\+T\+S\+\_\+\+E\+X\+I\+ST is defined, then two lock bits are available to the user and 3 mode settings are defined for these two bits.

If \+\_\+\+\_\+\+B\+O\+O\+T\+\_\+\+L\+O\+C\+K\+\_\+\+B\+I\+T\+S\+\_\+0\+\_\+\+E\+X\+I\+ST is defined, then the two B\+L\+B0 lock bits are available to the user and 4 mode settings are defined for these two bits.

If \+\_\+\+\_\+\+B\+O\+O\+T\+\_\+\+L\+O\+C\+K\+\_\+\+B\+I\+T\+S\+\_\+1\+\_\+\+E\+X\+I\+ST is defined, then the two B\+L\+B1 lock bits are available to the user and 4 mode settings are defined for these two bits.

If \+\_\+\+\_\+\+B\+O\+O\+T\+\_\+\+L\+O\+C\+K\+\_\+\+A\+P\+P\+L\+I\+C\+A\+T\+I\+O\+N\+\_\+\+T\+A\+B\+L\+E\+\_\+\+B\+I\+T\+S\+\_\+\+E\+X\+I\+ST is defined then two lock bits are available to set the locking mode for the Application Table Section (which is used in the X\+M\+E\+GA family).

If \+\_\+\+\_\+\+B\+O\+O\+T\+\_\+\+L\+O\+C\+K\+\_\+\+A\+P\+P\+L\+I\+C\+A\+T\+I\+O\+N\+\_\+\+B\+I\+T\+S\+\_\+\+E\+X\+I\+ST is defined then two lock bits are available to set the locking mode for the Application Section (which is used in the X\+M\+E\+GA family).

If \+\_\+\+\_\+\+B\+O\+O\+T\+\_\+\+L\+O\+C\+K\+\_\+\+B\+O\+O\+T\+\_\+\+B\+I\+T\+S\+\_\+\+E\+X\+I\+ST is defined then two lock bits are available to set the locking mode for the Boot Loader Section (which is used in the X\+M\+E\+GA family).

The A\+VR lockbit modes have inverted values, logical 1 for an unprogrammed (disabled) bit and logical 0 for a programmed (enabled) bit. The defined macros for each individual lock bit represent this in their definition by a bit-\/wise inversion of a mask. For example, the L\+B\+\_\+\+M\+O\+D\+E\+\_\+3 macro is defined as\+: 
\begin{DoxyCode}
\textcolor{preprocessor}{    #define LB\_MODE\_3  (0xFC)}
\textcolor{preprocessor}{`   }
\end{DoxyCode}
 \begin{DoxyVerb}To combine the lockbit mode macros together to represent a whole byte,
use the bitwise AND operator, like so:
\end{DoxyVerb}
 
\begin{DoxyCode}
(LB\_MODE\_3 & BLB0\_MODE\_2)
\end{DoxyCode}


$<$\hyperlink{lock_8h}{avr/lock.\+h}$>$ also defines a macro that provides a default lockbit value\+: L\+O\+C\+K\+B\+I\+T\+S\+\_\+\+D\+E\+F\+A\+U\+LT which is defined to be 0x\+FF.

See the A\+VR device specific datasheet for more details about these lock bits and the available mode settings.

A convenience macro, L\+O\+C\+K\+M\+EM, is defined as a G\+CC attribute for a custom-\/named section of \char`\"{}.\+lock\char`\"{}.

A convenience macro, L\+O\+C\+K\+B\+I\+TS, is defined that declares a variable, \+\_\+\+\_\+lock, of type unsigned char with the attribute defined by L\+O\+C\+K\+M\+EM. This variable allows the end user to easily set the lockbit data.

\begin{DoxyNote}{Note}
If a device-\/specific I/O header file has previously defined L\+O\+C\+K\+M\+EM, then L\+O\+C\+K\+M\+EM is not redefined. If a device-\/specific I/O header file has previously defined L\+O\+C\+K\+B\+I\+TS, then L\+O\+C\+K\+B\+I\+TS is not redefined. L\+O\+C\+K\+B\+I\+TS is currently known to be defined in the I/O header files for the X\+M\+E\+GA devices.
\end{DoxyNote}
\begin{DoxyParagraph}{A\+PI Usage Example}

\end{DoxyParagraph}
Putting all of this together is easy\+:


\begin{DoxyCode}
\textcolor{preprocessor}{#include <\hyperlink{io_8h}{avr/io.h}>}

LOCKBITS = (LB\_MODE\_1 & BLB0\_MODE\_3 & BLB1\_MODE\_4);

\textcolor{keywordtype}{int} main(\textcolor{keywordtype}{void})
\{
    \textcolor{keywordflow}{return} 0;
\}
\end{DoxyCode}


Or\+:


\begin{DoxyCode}
\textcolor{preprocessor}{#include <\hyperlink{io_8h}{avr/io.h}>}

\textcolor{keywordtype}{unsigned} \textcolor{keywordtype}{char} \_\_lock \hyperlink{stdint_8h_a772744ca0816d59e120b8f8a1ede64f0}{\_\_attribute\_\_}((section (\textcolor{stringliteral}{".lock"}))) = 
    (LB\_MODE\_1 & BLB0\_MODE\_3 & BLB1\_MODE\_4);

\textcolor{keywordtype}{int} main(\textcolor{keywordtype}{void})
\{
    \textcolor{keywordflow}{return} 0;
\}
\end{DoxyCode}


\begin{DoxyVerb}However there are a number of caveats that you need to be aware of to
use this API properly.

Be sure to include <avr/io.h> to get all of the definitions for the API.
The LOCKBITS macro defines a global variable to store the lockbit data. This 
variable is assigned to its own linker section. Assign the desired lockbit 
values immediately in the variable initialization.

The .lock section in the ELF file will get its values from the initial 
variable assignment ONLY. This means that you can NOT assign values to 
this variable in functions and the new values will not be put into the
ELF .lock section.

The global variable is declared in the LOCKBITS macro has two leading 
underscores, which means that it is reserved for the "implementation",
meaning the library, so it will not conflict with a user-named variable.

You must initialize the lockbit variable to some meaningful value, even
if it is the default value. This is because the lockbits default to a 
logical 1, meaning unprogrammed. Normal uninitialized data defaults to all 
locgial zeros. So it is vital that all lockbits are initialized, even with 
default data. If they are not, then the lockbits may not programmed to the 
desired settings and can possibly put your device into an unrecoverable 
state.

Be sure to have the -mmcu=<em>device</em> flag in your compile command line and
your linker command line to have the correct device selected and to have 
the correct I/O header file included when you include <avr/io.h>.

You can print out the contents of the .lock section in the ELF file by
using this command line:
\end{DoxyVerb}
 
\begin{DoxyCode}
avr-objdump -s -j .lock <ELF file>
\end{DoxyCode}
 