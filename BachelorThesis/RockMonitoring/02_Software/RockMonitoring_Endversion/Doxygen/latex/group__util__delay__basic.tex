\hypertarget{group__util__delay__basic}{}\section{$<$util/delay\+\_\+basic.h$>$\+: Basic busy-\/wait delay loops}
\label{group__util__delay__basic}\index{$<$util/delay\+\_\+basic.\+h$>$\+: Basic busy-\/wait delay loops@{$<$util/delay\+\_\+basic.\+h$>$\+: Basic busy-\/wait delay loops}}


\subsection{Detailed Description}

\begin{DoxyCode}
\textcolor{preprocessor}{#include <\hyperlink{delay__basic_8h}{util/delay\_basic.h}>}
\end{DoxyCode}


The functions in this header file implement simple delay loops that perform a busy-\/waiting. They are typically used to facilitate short delays in the program execution. They are implemented as count-\/down loops with a well-\/known C\+PU cycle count per loop iteration. As such, no other processing can occur simultaneously. It should be kept in mind that the functions described here do not disable interrupts.

In general, for long delays, the use of hardware timers is much preferrable, as they free the C\+PU, and allow for concurrent processing of other events while the timer is running. However, in particular for very short delays, the overhead of setting up a hardware timer is too much compared to the overall delay time.

Two inline functions are provided for the actual delay algorithms. 